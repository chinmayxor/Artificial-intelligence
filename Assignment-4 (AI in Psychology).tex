\documentclass[a4paper]{article} 
\input{head}
\begin{document}

%-------------------------------
%	TITLE SECTION
%-------------------------------

\fancyhead[C]{}
\hrule \medskip % Upper rule
\begin{minipage}{0.295\textwidth} 
\raggedright
\footnotesize
CHINMAY JAIN \hfill\\   
19111022\hfill\\
chinmaydhariwal1812@gmail.com
\end{minipage}
\begin{minipage}{0.4\textwidth} 
\centering 
\large 
Homework Assignment 4\\ 
\normalsize 

\end{minipage}
\begin{minipage}{0.295\textwidth} 
\raggedleft
%\today\hfill\\
\end{minipage}
\medskip\hrule 
\bigskip

%-------------------------------
%	CONTENTS
%------------------------------

%------------------------------------------------

\centering  \large  AI in Psychology \bigskip \hrule

\section{Attachment therapy}
Common name for a set of potentially fatal clinical interventions and parenting techniques aimed at controlling aggressive, disobedient, or unaffectionate children using "restraint and physical and psychological abuse to seek their desired results
Probably the most common form is holding therapy, in which the child is restrained by adults for the purpose of supposed cathartic release of suppressed rage and regression. Perhaps the most extreme, but much less common, is "rebirthing", in which the child is wrapped tightly in a blanket and then made to simulate emergence from a birth canal. This is done by encouraging the child to struggle and pushing and squeezing him/her to mimic contractions.
%------------------------------------------------

\section{Coding}
 It is a catch-all term for various Russian alternative therapeutic methods used to treat addictions, in which the therapist attempts to scare patients into abstinence from a substance they are addicted to by convincing them that they will be harmed or killed if they use it again. Each method involves the therapist pretending to insert a "code" into patients' brains that will ostensibly provoke a strong adverse reaction should it come into contact with the addictive substance. The methods use a combination of theatrics, hypnosis, placebos, and drugs with temporary adverse effects to instill the erroneous beliefs. Therapists may pretend to "code" patients for a fixed length of time, such as five years.

\section{Hypnosis}
It is state of extreme relaxation and inner focus in which a person is unusually responsive to suggestions made by the hypnotist. The modern practice has its roots in the idea of animal magnetism, or mesmerism.
Mesmer's explanations were thoroughly discredited, and to this day there is no agreement amongst researchers whether hypnosis is a real phenomenon, or merely a form of participatory role-enactment.
Some aspects of suggestion have been clinically useful.


\section{Memetics}

  It is an approach to evolutionary models of cultural information transfer based on the concept that units of information, or "memes", have an independent existence, are self-replicating, and are subject to selective evolution through environmental forces. Starting from a proposition put forward in the writings of Richard Dawkins, it has since turned into a new area of study, one that looks at the self-replicating units of culture. It has been proposed that just as memes are analogous to genes, memetics is analogous to genetics.

\end{document}
